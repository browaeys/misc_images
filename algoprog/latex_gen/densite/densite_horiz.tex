\documentclass[border=0pt]{standalone}
% pour du pdf

%\documentclass[dvisvgm,border=0pt]{standalone}
% pour du svg, en décommentant le latexmkr
 
\usepackage[utf8]{inputenc}
\usepackage[T1]{fontenc}
\usepackage{pgfplots}
\pgfplotsset{compat=1.18}
\usetikzlibrary{arrows.meta, calc}

% \pgfmathparse{2.718^(-#1 ^2 / 2)}
% %
% Définition de la fonction de densité
\pgfmathdeclarefunction{densite}{1}{%
  \pgfmathparse{1/(1+(#1/0.7)^2)}%
}

% Le code latexmkrc force une sortie en svg, même si c'est .pdf comme extension de fichier.

% Recompile :

% L'aperçu à droite affichera une erreur. C'est normal car Overleaf essaie d'ouvrir un fichier SVG comme s'il s'agissait d'un PDF.

% Téléchargez le fichier "PDF" via le bouton de téléchargement habituel.

% L'astuce finale : Une fois le fichier sur votre ordinateur, changez simplement son extension de .pdf en .svg.


\begin{document}

\begin{tikzpicture}

% --- GRAPHIQUE DE GAUCHE (F(x)) ---
\begin{axis}[
    name=leftplot,
    width=9cm, height=5cm,
    axis lines=middle,
    axis on top, 
    ymin=0, ymax=1.1, % Axe Y allongé
    xmin=-5, xmax=6,   % Axe X allongé
    xtick={-4, -2, 0, 2, 4}, xticklabels={},
    extra x ticks={0.8}, 
    extra x tick labels={$\vphantom{bp}x$}, 
    extra x tick style={
        tick label style={anchor=north, yshift=-2pt},
        tick style={black, thin} % Force le noir et l'épaisseur
    },
    ytick={0.2, 0.4, 0.6, 0.8, 1.0}, yticklabels={},
    tick align=center,
    axis line style={-{Stealth}},
    clip=false
]
    \addplot [fill=blue!20, draw=none, domain=-5:0.8, samples=500] {densite(x)} \closedcycle;
    \addplot [thick, black, samples=1000, domain=-5:5.2] {densite(x)} 
        node[right, pos=0.7, xshift=8pt, yshift=5pt] {$f(t)$}; % f(t) repositionné
    
    \draw [thin] (axis cs:0.8, 0) -- (axis cs:0.8, {densite(0.8)});
    
    \node (eq1) at (axis cs:-1.5, 1.3) {$\displaystyle F(x) = P(X \leq x) = \int_{-\infty}^{x} f(t) \, dt$};
    \draw [-{Stealth}, thin] (eq1.south) -- (axis cs:-1.0, 0.15); 
\end{axis}

% --- GRAPHIQUE DE DROITE (P(a < X < b)) ---
\begin{axis}[
    at={($(leftplot.east)+(1cm,0)$)}, 
    anchor=west,
    width=9cm, height=5cm,
    axis lines=middle,
    axis on top,
    ymin=0, ymax=1.1, % Axe Y allongé
    xmin=-5, xmax=6,   % Axe X allongé
    xtick={-4, -2, 0, 2, 4}, xticklabels={},
    extra x ticks={-0.8, 1.2}, 
    extra x tick labels={$\vphantom{bp}a$, $\vphantom{bp}b$},
    extra x tick style={
        tick label style={anchor=north, yshift=-2pt},
        tick style={black, thin} % Force le noir et l'épaisseur
    },
    ytick={0.2, 0.4, 0.6, 0.8, 1.0}, yticklabels={},
    tick align=center,
    axis line style={-{Stealth}},
    clip=false
]
    \addplot [fill=blue!20, draw=none, domain=-0.8:1.2, samples=500] {densite(x)} \closedcycle;
    \addplot [thick, black, samples=1000, domain=-5:5.2] {densite(x)} 
        node[right, pos=0.7, xshift=8pt, yshift=5pt] {$f(t)$}; % f(t) repositionné
    
    \draw [thin] (axis cs:-0.8, 0) -- (axis cs:-0.8, {densite(-0.8)});
    \draw [thin] (axis cs:1.2, 0) -- (axis cs:1.2, {densite(1.2)});
    
    \node (eq2) at (axis cs:-1, 1.3) {$\displaystyle P(a < X \leq b) = F(b) - F(a) = \int_{a}^{b} f(t) \, dt$};
    \draw [-{Stealth}, thin] (eq2.south) -- (axis cs:-0.3, 0.4);
\end{axis}

\end{tikzpicture}

\end{document}